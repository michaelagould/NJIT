\documentclass[10pt]{article}

\begin{document}

\title{Homework 3 - Proofs}
\author{Michael Gould\\ 
CS 506 - Online Foundations of CS}

\maketitle

1. Use \textbf{contra-positive proof} method for each of the following.

(a) There are 10 boxes.  Prove that if 40 balls are placed in the boxes, then at least one box has four or more balls.\\
\textbf{Contrapositive:}There are 10 boxes. If no box has more than 4 balls, then there are not 40 balls.\\
If a box exists with less that 4 balls, then there exists a box that contains $x$ balls where: 
$$x < 4$$\\
Suppose that all boxes have the maximum 3 balls fulfilling the $x < 4$ assumption
$$10 * 3 = 30 < 40$$\\ 
In which case total number of balls would in fact be less than that of 40.

(b) Let $x$ be a real number.  Prove that if $x^2$ is irrational, then $x$ must be irrational.\\
\textbf{Contrapositive:}Let $x$ be a real number. If $x$ is rational, then $x^2$ must be rational.  If $x$ is rational, then, given that $i$ and $j$ are integers, it can be represented by the expression 
$$\frac{i}{j}$$\\
Knowing that $x = \frac{i}{j}$ then, 
$$(\frac{i}{f})^2 = \frac{i^2}{j^2}$$\\
Proving that $x^2$ is in fact rational.

2. Use \textbf{contrapositive proof} for each of the following, where the domain of $n$ is positive integers.

(a) Prove that if $n^2$ is not divisible by 3, then $n$ is not divisible by 3.\\
\textbf{Contrapositive:} If $n$ is divisible by 3, then $n^2$ is divisible by 3.\\
If $n$ is divisible by 3, then it can be represented by the equation
$$3k$$
Where $k$ is an integer.\\ 
Knowing that $n = 3k$ with substitution we can conclude that:
$$(3k)^2 = 9k^2 = 3(3k^2)$$
This means that $n^2$ is divisible by 3.

(b) Prove that if $n^2$ is divisible by 3, then $n$ is divisible by 3. (Hint: If $n$ is not divisible by 3, then $n = 3k +r$, where $k$ is an integer quotient and $r$ is a non-zero remainder, $r \in{1,2} $.)\\
\textbf{Contrapositive:} If $n$ is not divisible by 3, then $n^2$ is not divisible by 3.\\
If $n$ is not divisible by 3, then
$$n = 3k + r$$ 
Where $k$ is an integer quotient and $r$ is a non-zero remainder $r \in{1,2}$.\\
To comprehend this in terms of $n^2$ we use substitution and get:
$$(3k +r)^2 = 9k^2 + 6kr + r^2 = 3(3k^2 + 2kr) + r$$
Dividing $n^2$ by 3 leaves us with $r \bmod 3$:
\center
\begin{tabular}{@{ }c@{ }@{ }|c | c@{ }@{ }c@{ }@{ }c@{ }@{ }c@{ }@{ }c}
$r$ & $r^2$ & $ r^2 \bmod 3 $ & & & & \\
\hline 
1 & 1 & 1 &  & \\
2 & 4 & 1 &  & \\
\end{tabular}
\flushleft
Which in all cases has no non-zero remainder, therefore $n^2$ is not divisible by 3.

3. Let $x$ and $y$ be two real numbers and let $A = (x + y)/2$.  We want to formally prove that if $(x > y)$ then
$$x < A < y$$
You are not allowed to state it as a known fact that the average of two values fall between those two values!  Rather, you must provide a formal proof in two ways:

(a) \textbf{Direct Method}

Where $A = (x + y)/2$ and $x < A < y$ using substitution we can assume the following:
$$x < (x + y)/2 < y$$
Using further math, we can manipulate the equation to find that:
$$2x < x + y < 2y$$
$$x < y < 2y - x$$
and thus it is shown that $x < y$

(b) \textbf{Contrapositive Method}.\\
Hints: For direct proof, assume $x < y$, and show that $2x < x + y < 2y$.\\
For contrapositive proof, assume $\neg(x < A < y)$, which means $\neg[(x < A)\wedge (A < y)]$, which is

$$(A \leq x) \vee (A \geq y)$$
Then provide the proof for each of the two cases in the OR statement.\\

In the case of $A \leq x$

$$(x + y)/2 \leq x$$
$$x + y \leq 2x$$
$$y \leq x$$

Or in the case of $A \geq y$

$$(x + y)/2 \geq y$$
$$x + y \geq y$$
$$x \geq y$$

Under both circumstances the negation is show to be equal and thus, the original statement is shown to be true. 

4. Use \textbf{proof by contradiction} for each of the following, where $x$ and $y$ are positive real numbers.

(a) Suppose $xy \geq 400$.  Prove that at least one of the two numbers must be $\geq 20$.

If $xy \geq 400$ then $x \vee y \geq 200$.  Suppose that $x$ OR $y < 20$ such that $xy \geq 400$.  The maximum value of $x$ that fulfills such a condition is 19, such that
$$19y \geq 400$$
$$y \geq 21.05...$$
Which proves false in the same manner that
$$19x \geq 400$$
$$x \geq 21.05$$
Which, again contradicts the above stated $x$ OR $y \geq 20$.\\

(b) Suppose $x$ is rational and $y$ is irrational.  Prove that $x * y$ is irrational.

Lets assume that $x * y$ is rational and thus can be represented by the following equation
$x * y = \frac{i}{j}$, where $i$ and $j$ are two integers.  We can also assumed by definition that $x$ can be represented in a similar fashion $x = \frac{p}{q}$  In which case,
$$y = (\frac{i}{j}) / (\frac{p}{q}) = (\frac{i}{j}) * (\frac{q}{p}) = \frac{iq}{pj}$$
Which contradicts the notion of $y$ being irrational, meaning the statement is true.\\

(c) Suppose $x$ is rational and $y$ is irrational.  Prove that $x + y$ is irrational.

Lets begin by assuming that $x + y$ is rational and thus can be represented by the equation $x + y = \frac{i}{j}$, where $i$ and $j$ are two integers.  $X$ is defined to be a rational as well and can be changed to the form $x = \frac{p}{q}$  in which case,
$$y = (\frac{i}{j}) - (\frac{p}{q}) = \frac{iq-jp}{jq}$$
Proving that $y$ by contradiction must be irrational.

5. Use proof by \textbf{contradiction} to show that $\sqrt{3}$ is irrational.
Hint: The proof is similar to the proof we did in class for $\sqrt{2}$.  Here, use the fact that if $n^2$ is divisible by 3, then $n$ is divisible by 3. (This was proven in one of the problems above.)\\
Assume that $\sqrt{3}$ is rational and can be represented by $\sqrt{3} = \frac{i}{j}$  We can also assume that the fraction $\frac{i}{j}$ in in lowest terms and has no common factors.  With that, we can manipulate this formula
$$ (\sqrt{3})^2 = (\frac{i}{j})^2$$ 
$$3 = (\frac{i}{j})^2$$ 
$$3 = (\frac{i^2}{j^2})$$
$$3j^2 = i^2$$
Earlier it was proven that given $n$ is divisible by 3, then $n^2$ is also divisible by 3, given the form
$n = 3k$, for some integer $n$.  Applying this principle we find that, $i^2 = 3k^2$ is in fact divisible by 3.  Further manipulations of the above equation reveal
$$3j^2 = i^2$$
$$(3j^2)/3 = (i^2)/3$$
$$j^2 = \frac{i^2}{3}$$
Knowing that $i^2$ is equivalent to $3k$ we find that
$$j^2 = \frac{(3k)^2}{3} = 3k^2$$
Which shows that $j^2 = 3k^2$ meaning that $j^2$ is divisible by 3.  Both $i$ and $j$ have a common factor of 3 which contradicts the assumption that $\frac{i^2}{j^2}$ have no common factors which means that $\sqrt{3}$ must be irrational.\\

6. Prove by induction that all integers for the following form are divisible by 4, for all integers $n \geq 1$.
$$f(n) = 5^n - 1$$

We begin by proving the base case, which in this case is when $n = 1$
$$f(1) = 5^1 - 1 = 4$$
We quickly find that the base case is true and move to find if $f(n+1)$ is also true.

$$f(n+1) = 5^{n+1} - 1$$
$$f(n+1) = 5 * 5^n - 1$$
$$f(n+1) = 5 * 5^n - 5 + 4$$
$$f(n+1) = 5(5^n - 1) + 4$$
By hypothesis $5^n - 1$ is divisible by 4, so the expression $4k$ appropriately substitutes in its stead.
$$f(n+1) = 5(4k) + 4 = 20k + 4$$
$$f(n+1) = 4(5k + 1)$$
In taking the form of $4k$ the proof via induction proves true.

7. Use \textbf{induction} to prove each of the following formulas.

(a)Arithmetic series sum:
$$S(n) = \sum\limits_{i=1}^n(i) = \frac{n(n +1)}{2}$$
$$S(n) = 1 + 2 + 3 + ... + n = \frac{n(n +1)}{2}$$
Base Case:
$$S(1) = 1 = \frac{2}{2} = 1$$
Proof of $S(n+1)$
$$S(n+1) = (1 + 2 + 3 + ... + n) + (n + 1) = \frac{(n + 1)(n + 2)}{2}$$
$$S(n+1) = S(n) + (n + 1) = \frac{(n + 1)(n + 2)}{2}$$
$$S(n+1) = \frac{n(n + 1)}{2} + (n + 1) = \frac{(n + 1)(n + 2)}{2}$$
$$S(n+1) = \frac{n(n + 1) + 2(n + 1)}{2} = \frac{(n + 1)(n + 2)}{2}$$
$$S(n+1) = \frac{(n + 1)(n + 2)}{2} = \frac{(n + 1)(n + 2)}{2}$$

(b)
$$ S(n) = \sum\limits_{i=1}^n(i^2) = \frac{n(n + 1)(2n + 1)}{6}$$
$$ S(n) = 1^2 + 2^2 + 3^2 + ... + n^2 = \frac{n(n + 1)(2n + 1)}{6}$$
Base Case:
$$ S(1) = 1^2 = 1 = \frac{(1)((1) + 1)(2(1) + 1)}{6} = \frac{6}{6} = 1$$
Proof of S(n+1)
$$ S(n+1) = (1^2 + 2^2 + 3^2 + ... + n^2) + (n+1)^2 = \frac{(n+1)((n+1) + 1)(2(n+1) + 1)}{6}$$
$$ S(n+1) = S(n) + (n+1)^2 = \frac{(n+1)(n + 2)(2(n+1) + 1)}{6}$$
$$ S(n+1) = \frac{n(n + 1)(2n + 1)}{6} + (n+1)^2 = \frac{(n+1)(n + 2)(2(n+1) + 1)}{6}$$
$$ S(n+1) = \frac{n(n + 1)(2n + 1)}{6} + (\frac{6}{6}((n+1)^2) = \frac{(n+1)(n + 2)(2(n+1) + 1)}{6}$$
$$ S(n+1) = \frac{n(n + 1)(2n + 1)+6(n+1)^2}{6} = \frac{(n+1)(n + 2)(2(n+1) + 1)}{6}$$
$$ S(n+1) = \frac{(n + 1)[n(2n + 1)+6(n+1)]}{6} = \frac{(n+1)(n + 2)(2(n+1) + 1)}{6}$$
$$ S(n+1) = \frac{(n + 1)(2n^2 + n + 6n + 6)}{6} = \frac{(n+1)(n + 2)(2(n+1) + 1)}{6}$$
$$ S(n+1) = \frac{(n + 1)(2n^2 + 7n + 6)}{6} = \frac{(n+1)(n + 2)(2(n+1) + 1)}{6}$$
$$ S(n+1) = \frac{(n + 1)(n + 2)(2(n + 1) + 1)}{6} = \frac{(n+1)(n + 2)(2(n+1) + 1)}{6}$$

(c) Geometric series sum, $a \neq 1:$
$$ S(n) = \sum\limits_{i=0}^n(a^i) = \frac{a^{n+1} - 1}{a - 1}$$
$$ S(n) = a^1 + a^2 + a^3 + ... + a^n = \frac{a^{n+1} - 1}{a - 1}$$
Base Case
$$ S(0) = a^0 = a = \frac{a^{1} - 1}{a - 1} = \frac{a-1}{a-1} = 1$$
Proof of S(n+1)
$$ S(n+1) = (a^1 + a^2 + a^3 + ... + a^n) + a^{n+1} = \frac{a^{(n+1)+1} - 1}{a - 1}$$
$$ S(n+1) = S(n) + a^{n+1} = \frac{a^{n+2} - 1}{a - 1}$$
$$ S(n+1) = \frac{a^{n+1} - 1}{a - 1} + a^{n+1} = \frac{a^{n+2} - 1}{a - 1}$$
$$ S(n+1) = \frac{a^{n+1} - 1}{a - 1} + (a^{n+1})(\frac{a-1}{a-1}) = \frac{a^{n+2} - 1}{a - 1}$$
$$ S(n+1) = \frac{a^{n+1} - 1 + (a-1)(a^{n+1})}{a - 1} +  = \frac{a^{n+2} - 1}{a - 1}$$
$$ S(n+1) = \frac{a^{n+1} - 1 + (a-1)(a^{n+1})}{a - 1} +  = \frac{a^{n+2} - 1}{a - 1}$$
$$ S(n+1) = \frac{a^{n+1} - 1 +(a^{n+1})(a-1)}{a - 1} +  = \frac{a^{n+2} - 1}{a - 1}$$
$$ S(n+1) = \frac{a^{n+1} - 1 + a^{n+2} - a^{n+1}}{a - 1} +  = \frac{a^{n+2} - 1}{a - 1}$$
$$ S(n+1) = \frac{a^{n+1} - 1}{a - 1} +  = \frac{a^{n+2} - 1}{a - 1}$$

8. A savings bank pays an interest rate of 5\%, compounded annually.  Consider an initial deposit of \$1000. Let $F_n$ be the total at the end of year $n$.  This function may be expressed recursively as follows:
$$F_0 = 1000$$
$$F_n = 1.05 * F_{n-1}, n \geq 1$$
(This recursive definition is called a \textit{recurrence equation}.)

(a) Compute $F_1,F_2,...,F_{10}$, and tabulate results (to get a feel for how the amount compounds).

\center
\begin{tabular}{@{ }c@{ }@{ }|c c@{ }@{ }c@{ }@{ }c@{ }@{ }c@{ }@{ }c}
& Output & & & & & \\
\hline 
$F(0)$ & 1000 & & & \\
$F(1)$ & 1050 & & & \\
$F(2)$ & 1102.50 & & & \\
$F(3)$ & 1157.63 & & & \\
$F(4)$ & 1215.51 & & & \\
$F(5)$ & 1276.28 & & & \\
$F(6)$ & 1340.10 & & & \\
$F(7)$ & 1407.10 & & & \\
$F(8)$ & 1477.46 & & & \\
$F(9)$ & 1551.33 & & & \\
$F(10)$ & 1628.90 & & & \\
\end{tabular}
\flushleft

(b) Prove by induction on $n$ that
$$F_n = 1000*1.05^n, n \geq 0$$
Base Case
$$F_0 = 1000*(1.05)^0, = 1000$$
Prove for $F_{n+1}$
$$F_{n+1} = (1.05)* F_n$$
$$F_{n+1} = (1.05)* (1000 * (1.05)^n)$$
$$F_{n+1} = 1000 * (1.05)^{n+1}$$
\end{document}
