\documentclass[10pt]{article}
\usepackage{algpseudocode}
\newcommand\tab[1][.5cm]{\hspace*{#1}}
\usepackage{amsmath}
\begin{document}

\title{Homework 6 - Algorithms}
\author{Michael Gould\\ 
CS 506 - Online Foundations of CS}

\maketitle

1. \textit{Prove the following polynomial is $\Theta(n^3)$.  That is, prove $T(n)$ is both $O(n^3)$ and $\Omega(n^3)$}

$$T(n) = 2n^3 - 10n^2 + 100n - 50$$

(a) \textit{Prove $T(n)$ is $O(n^3)$: By definition, you must find positibe constants $C_1$ and $n_0$ such that}

$$T(n) \leq C_1n^3, \forall n \geq n_0$$

(b) \textit{Prove $T(n)$ is $\Omega(n^3)$: By definition, you must find positibe constants $C_1$ and $n_0$ such that}

$$T(n) \geq C_1n^3, \forall n \geq n_0$$

2. (a) \textit{Compute and tabulate the following functions for $ n=1,2,4,8,16,32,64$.}

(b) \textit{Order the following complexity functions (growth rates) form the smallest to the largest.}

$$n^2 log n, 5, n log^2 n, 2^n, N^2, N, \sqrt{n} log n, \frac{n}{log n}$$

3. \textit{Find the exact number of times (in terms of $n$) the innnermost statement $(X=X-1)$ is executed in the following code.  That is, find the value of X, then express the total running time in terms of $O(),\Omega(),\Theta()$ as appropriate}

$X = 0;$\\
  \tab for $k=1$ to $n$\\
  \tab \tab for $j=1$ to $n-k$\\
  \tab \tab \tab $X=X+1;$\\

4. \textit{the following program computes and returns $(log_2n)$, assuming the input $n$ is an integer power of 2.  That is $n=2^j$ for some integer $j\geq0$}\\
\tab int LOG (int $n$)\{\\
\tab int $m,k$;\\
\tab $m=n$;\\
\tab $k=0$;\\
\tab while ($m>1$) \{\\
\tab \tab $m=m/2$;\\
\tab \tab $k=k+1$; \}\\
\tab return ($k$)\\
\tab \}

(a) \textit{First trace the execution of this program for a specific input value, $n=16$.  Tabulate the values of $m$ and $k$ at the beginning, just before the first execution of the while loop and after each execution of the while loop.}

(b) \textit{Prove by induction that at the end of each execution of the while loop, the following relation holds between variables $m$ and $k$.
}
$$m=\frac{n}{2^k}$$

(c) \textit{Then conclude that at the end, after the last iteration of the while loop, the program returns $k=log_2n$.}

5.\textit{The following pseudocode computes the sum of an array of $n$ integers}\\
\tab int sum (int $A[]$, int $n$) \{\\
\tab $T=A[0]$;\\
\tab for $i=1$ to $n-1$\\
\tab \tab $T=T+A[i]$;\\
\tab return $T$;\\
\tab \}

(a) \textit{Write a recursive version of this code}

(b) \textit{Let $f(n)$ be the number of additions performed by this computation.  Write a recurrance equation for $f(n)$.}

(c) \textit{Prove by induction that the solution of the recurrance is $f(n)=n-1$.}

6. \textit{The following pseudocode finds the maximum element in an array of size n.}\\
\tab int MAX (int $A[]$, int $n$)\{\\
\tab $M = A[0]$;\\
\tab for $i=1$ to $n-1$\\
\tab \tab if $(A[i] > M)$\\
\tab \tab \tab $M=A[i]$ //Update the max\\
\tab return $M$;
\tab \}

(a) \textit{Write a recursive version of this code}

(b) \textit{Let $f(n)$ be the number of key comparisons performed by this algorithm.  Write a recurrance equation for $f(n)$.}

(c) \textit{Prove by induction that the solution of the recurrance is $f(n)=n-1$.}

7. \textit{Consider the following pseudocode for insertion-sort algorithm.  The algorithm sorts an arbitrary array $A[0..n-1]$ of $n$ elements}\\
\tab void ISORT(dtype $A[]$, int $n$) \{\\
\tab int $i,j$;\\
\tab for $i=1$ to $n-1$ \{\\
\tab \tab //Insert $A[i]$ in the sorted part $A[0...i-1]$\\
\tab \tab $j=i$;\\
\tab \tab while $(j > 0$ and $A[j] < A[j-1])$ \{\\
\tab \tab \tab SWAP$(A[j],A[j-1])$ \{\\
\tab \tab \tab $j=j-1$ \}\\
\tab \tab \tab \}\\
\tab \tab \}\\
\tab \}\\

(a) \textit{Illustrate the algorithm on the following array by showing each comparison/swap operation.  What is the total number of comparisons made for this worst-case data?}

$$A=(5,4,3,2,1)$$

(b) \textit{Write a recursive version of this algorithm.}

(c) \textit{Let $f(n)$ be the worst-case number of key comparisons made bt this algorithm to sort $n$ elements.  Write a recurrance equation for $f(n)$}

(d) \textit{Find the solution for $f(n)$ by repeated substition}

8.\textit{Consider the bubble-sort algorithm described below.}\\
\tab void bubble (dtype $A[]$, int $n$) \{\\
\tab int $i,j$;\\
\tab for $(i=n-1$; $i>0$; $i--)$\\
\tab \tab for$(j=0$; $j<i$; $j++)$\\
\tab \tab \tab if$(A[j]>A[j+1])$ \tab SWAP($A[j],A[j+1])$;
\tab \}

(a) \textit{Analyze the time complexity, $T(n)$, of the bubble-sort algorithm.}

(b) \textit{Rewrite the algorithm using recursion}.

(c) \textit{Let $f(n)$ be the worst-case number of key-comparisons used by this algorithm to sort $n$ elements.  Write a recurrance for $f(n)$.  Solve the recurrance by repeated subsititions}

9. \textit{The wollowing algorithm uses a} \textbf{divide-and-conquer} \textit{technique to find the maximum element in an array of size $n$.  The initial call to this recursive function is max(arrayname, 0, n).}\\
\tab dtype Findmax(dtype $A[]$, int $i$, int $n$)\\
\tab \{\tab //$i$ is the starting index and $n$ is the number of elements\\
\tab dtype $Max1,Max2$;\\
\tab if$(n==1)$ return $A[i]$;\\
\tab $Max1$ = Findmax$(A,i,\left \lfloor{n/2}\right \rfloor)$;\tab \tab \tab //Find max of the first half\\
\tab $Max2$ = Findmax$(A,i+\left \lfloor{n/2}\right \rfloor,\left \lceil{n/2}\right \rceil)$;\tab //Find max of the second half\\
\tab if $(Max1 \geq Max2)$ \tab return $Max1$;\\
\tab \tab else return $Max2$;\\
\tab \}

\textit{Let $f(n)$ be the worst-case number of key comparisons for finding the max of $n$ elements}

(a) \textit{Assuming $n$ is a power of 2, write a recurrance relation for $f(n)$.  Find the solution by each of the following methods.}

\tab (i) \textit{Apply the repeated substition method.}

\tab (ii) \textit{Apply induction to prove that $f(n)=An+B$ and find the constants $A$ and $B$.}

(b) \textit{Now consider the general case where $n$ is any integer.  Write a recurrance for $f(n)$.  Use induction to prove that the somution is $f(n)=n-1$}.

10. \textit{The following divide-and-conquer algorithm is designed to return TRUE if and only if all the elements of the array have equal values.  For simplicity, suppose the array size is $n=2^k$ for some integer $k$.  Input $S$ is the starting index and $n$ is the number of elements starting at $S$.  The initial call is SAME$(A,0,n)$.}\\
\tab Boolean SAME (Int $A[]$, int $S$, int $n$) \{\\
\tab Boolean $T1,T2,T3$;\\
\tab if $(n==1)$ return TRUE;\\
\tab T1 = SAME $(A,S,n/2)$;\\
\tab T2 = SAME $(A,S+n/2,n/2)$;\\
\tab T3 = $(A[S] == A[S+n/2])$;\\
\tab return $(T1 \wedge T2 \wedge T3)$;\\
\tab \}\\
(a) \textit{Explain how this program works}

(b) \textit{Prove by induction that the algorithm returns TRUE if and only if all the elements of the array have equal values.}

(c) \textit{Let $f(n)$ be the number of key comparisons in this algorithm for any array of size $n$.  Write a Recurrance for $f(n)$.}

(d) \textit{Find the solutio by repeated subsitition.}

\end{document}