\documentclass[10pt]{article}
\usepackage{amsmath}
\begin{document}

\title{Homework 4 - Functions}
\author{Michael Gould\\ 
CS 506 - Online Foundations of CS}

\maketitle

1. Prove by \textbf{simple induction} that every postage amount of 8 cents or more can be achieved by using only 5-cent stamps and 3-cent stamps.  That is, prove for every integer 8 $\geq$ n there exists non-negative integers $A$ and $B$ such that:

$$n = 5A + 3B$$

Base case, where $n = 8$
$$8 = 5(1) + 3(1)$$

Which is in fact true, now we apply the circumstance under which the input is $n + 1$

$$n+1 = 5A' + 3B'$$

To complete the equality of this equation, we must manipulate $A$ and $B$ as such that the net gain of 1 is equivalent on both sides by means of subtracting 1 on the opposing side. Thus there is the necessity to find two variables whose products satisfy these conditions with the given coefficients.  This can be done in two ways:

$$n+1 = 5(A+2) + 3(B-3), B \geq 3$$
AND
$$n+1 = 5(A-1) + 3(B+2), A \geq 1$$

Given this information, there will always be a case in which one of these two circumstances will apply.  This can be proven by means of contradiction, assuming that we take values that fulfill neither $A$ nor $B$, such as $(A = 0 \wedge B \leq 2)$.  If we do, the $n$ produced by these inputs would always produce $n \leq 6$ which is defined as false by the bounds of the initial definition $n \geq 8$


2. The following "proof by induction" attempts to prove that all horses in universe have the same color.  That is. for every $n \geq 1$, any set of $n$ horses in universe have the same color.  Obviously, there is something wrong with this proof.  State clearly which step is wrong and why.

Proof:

(a) For $n = 1$, the set contains a single horse, which has the same color by itself, so the base case is obviously correct\\
(b) For the hypothesis, suppose the claim is correct for some $n \geq 1$.\\
(c)Then, we will prove the claim is also correct for $n + 1$.  Consider any set of $n + 1$ horses.  Let us number them as 1,2,...,$n+1$.\\
i. By the hypothesis, horses ${1,2,...n}$ have the same color.\\
ii. By the hypothesis, horses ${2,3,...n+1}$ have the same color.\\
iii. These two sets have in common horses ${2,3,...n}$.\\
iv. Therefore by transitivity, these $n + 1$ horses all have the same color.\\

This hypothesis is mathematically sound, but logically flawed. Through the use of induction, one can in fact prove that he base case is true, leaving us to believe that the use of induction is feasible, but each new iteration of the induction steps omits the previous horse from the initiate set.  By then finding the following set to have the same properties (that the $n+1$ has the same color) you are proving that each set continues to have one color, indiscriminate of the original horse's color.

3. Proof by \textbf{induction} that every postage of 24 cents or more can be achieved by using only 7-cent stamps and 5-cent stamps.  That is, prove that for every integer $n \geq 24$, there exists some \textbf{non-negative} integers $A$ and $B$ such that
$$P(n): n = 7A + 5B$$
Let $P(n)$ be the predicate "postage of $n$ cents can be achieved." Provide the inductive proof by using each of the following methods.\\
(a)(\textbf{Simple Induction}) Prove the induction base $n = 24$.  Then for any $n \geq 24$, prove that if $P(n)$ is true, then $P(n + 1)$ will be true.\\
Given the base case of $n = 24$ we find that:

$$24 = 7(2) + 5(2)$$

Which is in fact true to the confines of the equation, now we change the situation in which the input is $n + 1$

$$n+1 = 7A' + 5B'$$

the equality of said equation based on the initiate $n$ situation would include the negation of the added value to the right side of the equation, by use of the variables $A$ and $B$.

$$n+1 = 7(A-2) + 5(B+3), A \geq 2$$
OR
$$n+1 = 7(A+3) + 5(B-4), B \geq 4$$

To prove the validity of either of these two statements, we can use a contradiction to prove its truth.  Assume that we are fulfilling neither $A \geq 1$ nor $B \geq 3$ by using the inputs $A = 1 \wedge B = 3$.  Using these two inputs we find that $23 \geq 24$ which contradicts the initial definition.

(b)(\textbf{Strong Induction}) Prove the induction base cases (24,25,...,30).  Then for $n \geq 31$, prove that:

\center
If $P(m)$ is true for all $24 \leq m \leq n$, then $P(n)$ is also true.\\
\flushleft



\center
\begin{tabular}{@{ }c@{ }@{ }|c c@{ }@{ }c@{ }@{ }c@{ }@{ }c@{ }@c}
P(n) & Output \\
\hline
$P(24)$ & 24 = 7(2) + 5(2) & & & \\
$P(25)$ & 25 = 7(0) + 5(5) & & & \\
$P(26)$ & 26 = 7(3) + 5(1) & & & \\
$P(27)$ & 27 = 7(1) + 5(4) & & & \\
$P(28)$ & 28 = 7(4) + 5(0) & & & \\
$P(29)$ & 29 = 7(2) + 5(3) & & & \\
$P(30)$ & 30 = 7(0) + 5(6) & & & \\
\end{tabular}
\flushleft

Since $n \geq 31$, then $n - 7 \geq 24$ so we know from the hypothesis that $P(n-7)$ is true.  Which provides us with

$$(n-7) = 7A + 5B$$

Given that the coefficient of the expression provides us the necessary equivalent augmentation by virtue of the $A$ variable, we can now express an equivalency by adding one to it, shown below.

$$n = 7(A+1) + 5B$$

Which proves the induction proof.

4. Prove each formula by induction\\
(a)
$$S(n) = 1 + 3 + 5 + ... + (2n - 1) = n^2$$

Base case:
$$S(0) = 2(1) - 1 = 1 = 1^2 = 1$$

Proof by Induction:
$$S(n+1) = (1 + 3 + 5 + ... + (2n - 1)) + (2(n+1)-1) = (n+1)^2$$
$$S(n+1) = S(n) + 2n+2-1 = (n+1)^2$$
$$S(n+1) = n^2 + 2n + 1 = (n+1)^2$$
$$S(n+1) = (n + 1)(n + 1) = (n+1)^2$$
$$S(n+1) = (n + 1)^2 = (n+1)^2$$

(b) 
$$\sum\limits_{i=1}^n (\frac{i}{2^i}) = 2 -\frac{n+2}{2^n}$$

Base case:
$$S(1) = \frac{1}{2^1} = 2 - \frac{1+2}{2^1}$$
$$S(1) = \frac{1}{2} =  \frac{3}{2} - 2$$
$$S(1) = \frac{1}{2} = \frac{3}{2} - \frac{2}{2}$$
$$S(1) = \frac{1}{2} = \frac{1}{2}$$

Proof by Induction:
$$S(n) = \frac{1}{2^1} + \frac{2}{2^2} + \frac{3}{2^3} + ... + \frac{n}{2^n} = 2 - \frac{n+2}{2^n}$$
$$S(n+1) = (\frac{1}{2^1} + \frac{2}{2^2} + \frac{3}{2^3} + ... + \frac{n}{2^n}) + \frac{n+1}{2^{n+1}} = 2 - \frac{(n+1)+2}{2^{n+1}}$$
$$S(n+1) = S(n) + \frac{n+1}{2^{n+1}} = 2 -\frac{n+3}{2^{n+1}}$$
$$S(n+1) = 2 - \frac{n+2}{2^n} + \frac{n+1}{2^{n+1}} = 2 -\frac{n+3}{2^{n+1}}$$
$$S(n+1) =  -(\frac{n+2}{2^n}) + \frac{n+1}{2^{n+1}} = -(\frac{n+3}{2^{n+1}})$$
$$S(n+1) =  -(\frac{2n+4}{2^{n+1}}) + \frac{n+1}{2^{n+1}} = -(\frac{n+3}{2^{n+1}})$$
$$S(n+1) =  -(\frac{2n+4}{2^{n+1}}) + \frac{n+1}{2^{n+1}} = -(\frac{n+3}{2^{n+1}})$$
$$S(n+1) =  \frac{-2n-4+n+1}{2^{n+1}} = -(\frac{n+3}{2^{n+1}})$$
$$S(n+1) =  \frac{-n-3}{2^{n+1}} = -(\frac{n+3}{2^{n+1}})$$
$$S(n+1) =  \frac{-n-3}{2^{n+1}} = \frac{-n-3}{2^{n+1}}$$

5. Prove by induction that for all integers $n \geq 1$.
$$2^n > n$$

Base case $n = 1$:
$$2^1 > 1 = 2 > 1$$

Proof by induction (where $n = n +1$)
$$2^{n+1} > n + 1$$
$$2^{n+1} = 2(2n) > n+1$$

Given the Hypothesis, as it is assumed to be true:

$$2^{n+1} = 2(2n) > n+1$$
And
\center
$2(n) \geq n + 1$, while $n \geq 1$
\flushleft
Thus,
$$2^{n+1} > n + 1$$

6. Prove by induction that the following pseudocode computed $X^n$. (This algorithm computes power repeated multiplications.) 
\center
$T$ = 1;\\
for $i$ = 1 to $n$ [ \\
$T = T * X$]
\flushleft
Hint: Use induction to probe that after iteration $i$ of the loops, where $i = 1,2,3,...,n$, the result is $T = X^i$.  (This is called a loop invariant.) Therefore, after the last iteration, $i = n$ and so $T = X^n$.

Consider the case of $n = 0$.  With the given pseudocode above, we can show that it computes $X^0$ accurately.  Initially $T$ is set to 1, the input 0 disallows the entrance into the for loop and thus $T$ contains the accurate computation of $X^0$ which is X.  We will consider this the base case.\\

We assume the pseudocode's equality to $X^n$ for an arbitrary $n$, namely $n=0$.  We can now show that the $n$-th case implies the $n+1$ case.  With the return of $T$ being equal to that of $X^n$, the $n$-th iteration of the loop with leave $T$ assigned to $X^n$. When the subsequent iteration is executed the pseudocode will attempt to assign $T$ to $T * X$.  Since $T = X^n$, we can subsitute $T$ for $X^n$ leaving the equation as $X^n * X$, which equals  $X^{n*1}$.\\
 
7. Prove by induction that the following pseudocode computes $X^n$, where $n = 2^k$. (this algorithm is much more efficient than the earlier one.  In the earlier algorithm, the power goes up by 1 after each iteration of the loop, but in this algorithm the power doubles after each iteration.)
\center
realPower(real $X$, int $k$)[\\
$T = X$;\\
for $i$ = 1 to $k$ [\\
$T = T * T$]\\
return (T)
]
\flushleft
Hint: Use induction to prove that after iteration $i = m$ of the look, $T = X^p$, where $p = 2^m$

Consider the case of $k = 0$.  With the given pseudocode above, we can show that it computes $X^{2^0}$, initially $T$ is set to $X$ the input 0 disallows the entrance into the for loop and thus $T$ contains the accurate computation of $X^{2^0}$ which is X.  We will consider this the base case.\\

We assume the pseudocode's equality to $X^{2^k}$ for an arbitrary $k$, namely $k=0$. We can now show that the $k$-th case implies the $k+1$ case.  With the return of $T$ being equal to that of $X^{2^K}$ the $k$-th iteration of the loop will leave $T$ assigned to $X^{2^k}$.  When the subsequent iteration is executed the pseudocode will attempt to assign $T$ to $T * T$, we can substitute $T$ for $X^{2^k}$ leaving the equation as $X^{2^k} * X^{2^k} = X^{2^{k+1}}$.  Given exponent laws and algebra, 

$$X^{2^k} * X^{2^k} = X^{2^{k+1}}$$
$$X^{2^k+2^k} = X^{2^{k+1}}$$
$$X^{2(2^k)} = X^{2^{k+1}}$$
$$X^{2^{k+1}} = X^{2^{k+1}}$$

8. Consider the following recurrence equation.\\
Note: The symbol $\left \lfloor{}\right \rfloor$ is the "floor" function.  For any real $x,\left \lfloor{x}\right \rfloor$ rounds down $x$ to its nearest integer.  For example, $\left \lfloor{3.1415}\right \rfloor = 3$, and $\left \lfloor{3}\right \rfloor = 3$

\begin{equation}
  f(x)=\begin{cases}
    1, & $n = 1$\\
    f(\left \lfloor{n/2}\right \rfloor + n, & n \geq 2
  \end{cases}
\end{equation}

(a) Compute and tabulate $f(n)$ for $n =$ 1 to 8.\\
\center
\begin{tabular}{@{ }c@{ }@{ }|c c@{ }@{ }c@{ }@{ }c@{ }@{ }c@{ }@c}
\hline
$f(1)$ & 1 & & & \\
$f(2)$ & 3 & & & \\
$f(3)$ & 4 & & & \\
$f(4)$ & 6 & & & \\
$f(5)$ & 7 & & & \\
$f(6)$ & 9 & & & \\
$f(7)$ & 10 & & & \\
$f(8)$ & 12 & & & \\
\end{tabular}
\flushleft
(b) Prove by induction that the solution has the following bound.
$$f(n) < 2n$$

\textbf{Hint:} A strong form of induction is needed here.  (Don't try to increment $n$ by 1 in your induction step.) To prove the bound for any $n$ you have to assume the bound is true for all smaller values of $n$.  That is, assume $f(m) < 2m$ for all $m<n$.  This strong hypothesis in particular will mean
$$f(\left \lfloor{n/2}\right \rfloor) < 2(\left \lfloor{n/2}\right \rfloor)$$

In terms of the base case where $x = 1$ we see that $f(x) = 1$, which follows the bound given $f(1) = 1 < 2(1) = 2$ proving it to be true.  To further prove that the bound is true for all $n \geq 2$ we suppose that it holds for all smaller values and thusly $f(m) < 2m$ for all values in which $m < n$ by hypothesis

$$f(\left \lfloor{n/2}\right \rfloor) < 2(\left \lfloor{n/2}\right \rfloor)$$
$$f(n) = f(\left \lfloor{n/2}\right \rfloor) < 2\left \lfloor{n/2}\right \rfloor < 2(n/2) + n = 2n$$


\end{document}