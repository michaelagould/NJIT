\documentclass[10pt]{article}
\usepackage{amsmath}
\newcommand\tab[1][1cm]{\hspace*{#1}}
\begin{document}

\title{Program 2 - Tower of Hanoi}
\author{Michael Gould\\ 
CS 506 - Online Foundations of CS}

\maketitle
\textit{First, write a recursive algortihm to solve the problem.}\\

Using pseudo-code, the recursive algorithm to solving the towers of hanoi is as follows:

move($n$, A, B, C){\\
\tab if(n = 1){\\
\tab\tab move disk $n$ from A to B\\
\tab\tab end\\
\tab\}\\
\tab else {\\
\tab\tab move(n-1, A, C, B)\\
\tab\tab move disk $n$ from A to B\\
\tab\tab move(n-1, C, B, A)\\
\tab\}\\

(See enclosed C program for specific implementation of the algorithm)

\textit{Write a recurrence equation for $f(n)$ and solve the
recurrence. To get a better appreciation for this time complexity, tabulate $f(n)$ for the values of $n$ from 1
to 20.}

\begin{equation}
  f(n)=\begin{cases}
    1, & $n = 1$\\
    2F_{n-1} + 1, & $n $>$ 1$\\
  \end{cases}
\end{equation}

\center
\begin{tabular}{@{ }c@{ }@{ }|c c@{ }@{ }c@{ }@{ }c@{ }@{ }c@{ }@c}
$f(n)$ & Output \\
\hline
$F1$ & 1 & & & \\
$F2$ & 2(1) + 1 = 3 & \\
$F3$ & 2(3) + 1 = 7 & & & \\
$F4$ & 2(7) + 1 = 15& & & \\
$F5$ & 2(15) + 1 = 31& & & \\
$F6$ & 2(31) + 1 = 63& & & \\
$F7$ & 2(63) + 1 = 127& & & \\
$F8$ & 2(127) + 1 = 255& & & \\
$F9$ & 2(255) + 1 = 511& & & \\
$F10$ & 2(511) + 1 = 1,023& & & \\
$F11$ & 2(1023) + 1 = 2,047 & & & \\
$F12$ & 2(2047) + 1 = 4,095 & \\
$F13$ & 2(4095) + 1 = 8,191 & & & \\
$F14$ & 2(8191) + 1 = 16,383 & & & \\
$F15$ & 2(16383) + 1 = 32,767 & & & \\
$F16$ & 2(32767) + 1 = 65,535 & & & \\
$F17$ & 2(65535) + 1 = 131,071 & & & \\
$F18$ & 2(131071) + 1 = 262,143 & & & \\
$F19$ & 2(262143) + 1 = 524,287 & & & \\
$F20$ & 2(524287) + 1 = 1,048,575 & & & \\
\end{tabular}
\flushleft

\textit{Solve the recurrance formula for f(n)}\\

To solve this recurrane formula, we can observe a pattern from the above shown outputs for the first 20 instances of $n$.  We notice that the growth is exponential and can assume that it also includes some constant as well, leaving us the guess:

$$f(n) = A2^n + b$$

Using the above guess as the hypothesis for our induction, we can start by using the base cases from the recurrance equation

$$f(1) = 1$$
$$f(1) = 2A + B$$

The first of the above equations is straight from the recurrance equation above, the latter is the solution form we are looking to achieve, therefore the goal is to find $2A + B = 1$

We begin by making our guess the hypothesis and assuming it is correct for some $n /geq 1$:

$$f(n) = A2^n + B$$

Then we must also prove the solution is also correct for $n+1$

$$f(n+1) = A2^{n+1} + B$$

To prove the conclusion, we can start with the recurrance equation with the input $n+1$ and use the hypothesis to substitute in for $f(n)$.  In addition to some distributive law as well we find

$$f(n+1) = 2f(n) + 1$$
$$f(n+1) = 2(A2^n + B) + 1$$
$$f(n+1) = A2^{n+1} + (2B + 1)$$
$$f(n+1) = A2^{n+1} + B$$

The leap in the last step is that of a term by term equaltiy, we will need this equality to help find the constant $B$ which is can now be used in conjunction with our initial equation

$$2B + 1 = B$$
$$2A + B = 1$$

Using algebra we find that $A = 1$ and $B = -1$ and thus we have the constants to properly fill in the previous guess turned hypothesis

$$f(n) = A2^n + B$$

becomes

$$f(n) = 2^n + 1$$

\textit{Run your program for two values of n.}\\

\center
\begin{tabular}{@{ }c@{ }@{ }|c c@{ }@{ }c@{ }@{ }c@{ }@{ }c@{ }@c}
4 Disks & 3 Disks \\
\hline
Tower of Hanoi Solver & Tower of Hanoi Solver\\ 
Input number of disks (max 6): 4 & Input number of disks (max 6): 3\\
1 Move disc 1 from A to C & 1 Move disc 1 from A to B\\
2 Move disk 2 from A to B & 2 Move disk 2 from A to C\\
3 Move disc 1 from C to B & 3 Move disc 1 from B to C\\
4 Move disk 3 from A to C & 4 Move disk 3 from A to B\\
5 Move disc 1 from B to A & 5 Move disc 1 from C to A\\
6 Move disk 2 from B to C & 6 Move disk 2 from C to B\\
7 Move disc 1 from A to C & 7 Move disc 1 from A to B\\
8 Move disk 4 from A to B &\\
9 Move disc 1 from C to B &\\
10 Move disk 2 from C to A &\\
11 Move disc 1 from B to A &\\
12 Move disk 3 from C to B &\\
13 Move disc 1 from A to C &\\
14 Move disk 2 from A to B &\\
15 Move disc 1 from C to B &\\
\end{tabular}
\flushleft

\end{document}